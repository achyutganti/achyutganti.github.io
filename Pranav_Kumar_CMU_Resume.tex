%%%%%%%%%%%%%%%%%%%%%%%%%%%%%%%%%%%%%%%%%
% Plasmati Clean
% LaTeX Template
%
% This template has been downloaded from:
% http://www.LaTeXTemplates.com
%
% Original author (Plasmati Graduate CV):
% Alessandro Plasmati (alessandro.plasmati@gmail.com)
%
% Typography fixes:
% Grant Wu (grantwwu@gmail.com)
% Pranav Kumar (pranavmk98@gmail.com)
%
% License:
% CC BY-NC-SA 3.0 (http://creativecommons.org/licenses/by-nc-sa/3.0/)
%
% Important note:
% This template needs to be compiled with XeLaTeX.
% This template still has a lot of cruft in it that I haven't bothered cleaning
% up.
%
%%%%%%%%%%%%%%%%%%%%%%%%%%%%%%%%%%%%%%%%%

\documentclass[letter,twoside,11pt]{article} % Default font size and paper size
\usepackage{fontspec} % For loading fonts
\usepackage{datetime2}
\defaultfontfeatures{Mapping=tex-text}
\usepackage{xunicode,xltxtra,url,parskip} % Formatting packages
\usepackage[usenames,dvipsnames]{xcolor} % Required for specifying custom colors
\usepackage[top=0.55in,bottom=0.55in,left=0.35in,right=0.35in]{geometry}
% To reduce the height of the top margin uncomment: \addtolength{\voffset}{-1.3cm}
\definecolor{linkcolour}{rgb}{0,0.2,0.6} % Link color
\usepackage{hyperref}
\usepackage{enumitem}
\usepackage{fancyhdr}
\hypersetup{colorlinks,breaklinks,urlcolor=linkcolour,linkcolor=linkcolour} % Set link colors throughout the document
\usepackage{titlesec} % Used to customize the \section command
\titleformat{\section}{\Large\scshape\raggedright}{}{0em}{}[\titlerule] % Text formatting of sections
\titlespacing{\section}{0pt}{0pt}{0pt} % Spacing around sections

% This spacer is necessary to get the text boxes to line up across sections
\def\spacer{
  \begin{tabular}[t]{@{}p{36em}} \hphantom{lorem ipsum} \end{tabular} &
  \begin{tabular}[t]{r@{}} \hphantom{\textsc{May 2021 (Antttttt)}} \end{tabular}
  \vspace{-2.5em} \\
}

\begin{document}

\pagestyle{fancy} % Removes page numbering
\renewcommand{\headrulewidth}{0pt}
\fancyhead{}
\fancyfoot{}
\fancyfoot[R] {Last updated \today}

\begin{center}
{\Huge Pranav Kumar} \\
\textsc{Phone:} 412-726-1606 \\
\textsc{Email:} \href{mailto:pmkumar@cmu.edu}{pmkumar@cmu.edu}\\
Citizenship: USA
\end{center}

\section{Education}
\begin{center}
  \begin{tabular}{lr}
  % \centering
    \spacer
    % \begin{tabular}[t]{r@{}}\textsc{May 2021} (Anticipated) \\ \\ \textbf{QPA: 4.00/4.00} \\ Dean's List \\ (All semesters) \end{tabular} &
    \begin{tabular}[t]{@{}p{36em}}
      \textbf{B.S. in Computer Science, Carnegie Mellon University} \\
      \textsc{Relevant Coursework:}

      \textbf{15-410} - Operating System Design and Implementation

      \textbf{15-411} - Compiler Design

      \textbf{15-330} - Introduction to Computer Security

      \textbf{15-213} - Introduction to Computer Systems

      % \textbf{15-251} - Great Theoretical Ideas in Computer Science
      
    %   \textbf{15-210} - Parallel and Sequential Data Structures and Algorithms

      % \textbf{15-150} - Principles of Functional Programming, 

      % \textbf{15-122} - Principles of Imperative Programming, 

      % \textbf{15-241} - Matrices and Linear Transformations
    \end{tabular} &
    \begin{tabular}[t]{r@{}}\textsc{May 2021} (Anticipated) \\ \textbf{QPA: 4.00/4.00} \\ Dean's List \\ (5/5 semesters) \end{tabular} \\
  \end{tabular}
\end{center}

\vspace{-1em}

\section{Work Experience}
\begin{center}
  \begin{tabular}{lr}
    \spacer

    % % Citadel
    % \begin{tabular}[t]{@{}p{36em}}
    %   \textbf{Incoming Software Engineering Intern} | \textbf{Citadel}
    %   \begin{itemize}
    %     \item
    %   \end{itemize}
    % \end{tabular} &
    % \begin{tabular}[t]{r@{}} \textsc{May - August 2020} \end{tabular} 
    % \vspace{-0.5em} \\

    % Facebook
    \begin{tabular}[t]{@{}p{36em}}
      \textbf{Production Engineering Intern} | \textbf{Facebook}
    %   \textbf{Facebook} | \textbf{Production Engineering Intern}
      \begin{itemize}[leftmargin=*]
        \item Developed a build automation service for regression detection and parallel building of binaries
        \item Enabled multiple asynchronous builds, improving firefighting efficiency by \textbf{more than 10x}
      \end{itemize}
    \end{tabular} &
    \begin{tabular}[t]{r@{}} \textsc{May - August 2019} \end{tabular} 
    \vspace{-0.5em} \\

    % Petuum
    \begin{tabular}[t]{@{}p{36em}}
      \textbf{Software Engineering Intern} | \textbf{Petuum Inc}
    % \textbf{Petuum Inc} | \textbf{Software Engineering Intern}
      \begin{itemize}[leftmargin=*]
        \item {Developed a backend datapull API using MariaDB and Django for ML engineers}
        \item {Used Apache Kafka to integrate real-time streaming data with ML model}
        \item {Improved model inference runtime from 30+ minutes to 90 seconds (\textbf{20x speedup)}}
      \end{itemize}
    \end{tabular} &
    \begin{tabular}[t]{r@{}} \textsc{June - August 2018} \end{tabular} 
    \vspace{-0.5em} \\

    % 122 TA
    \begin{tabular}[t]{@{}p{36em}}
      \textbf{Lead Teaching Assistant} | \textbf{15-122}: Principles of Imperative Computation
      \begin{itemize}[leftmargin=*]
        % \item Taught fundamental data structures and algorithms concepts to 75\% of incoming CMU CS students
        % \item Taught labs, graded homeworks and hosted office hours for students
        \item Responsible for leading a course of 480+ students and 40+ teaching assistants
        \item Teach data structures and algorithms to Carnegie Mellon CS undergraduates
      \end{itemize}
    \end{tabular} &
    \begin{tabular}[t]{r@{}} \textsc{Spring 2018 - Present} \end{tabular} 
    \vspace{-0.5em} \\

    % Cerner
    % \begin{tabular}[t]{@{}p{36em}}
    %   \textbf{Intern} | \textbf{Cerner Corporation}
    %   \begin{itemize}[leftmargin=*]
    %     \item Synced client and server data with PHP and JavaScript for SmartHealth
    %     \item Laid foundation for internationalization using JSON
    %   \end{itemize}
    % \end{tabular} &
    % \begin{tabular}[t]{r@{}} \textsc{May 2016} \end{tabular}
  %   \vspace{-0.5em} \\
    % \begin{tabular}[t]{r@{}} \textsc{2014 - 2015} \end{tabular} &
    % \begin{tabular}[t]{@{}p{33em}}
    %   \textbf{Student Developer at Mediawiki Foundation}
    %   \begin{itemize}[leftmargin=*]
    %     \item Worked on open source applications such as Wikipedia, Mediawiki, etc.
    %     \item Learned Git, PHP, and the practice of code review in the process
    %   \end{itemize}
    % \end{tabular}
    % \vspace{-0.5em}
  \end{tabular}
\end{center}

\section{Projects and Extracurriculars}
\begin{center}
  \begin{tabular}{lr}
    \spacer

    % PennApps
    \begin{tabular}[t]{@{}p{36em}}
      \textbf{No Duckling is Ugly} | \textbf{PennApps 2018}
      \begin{itemize}[leftmargin=*]
        \item An IoT based anti-bullying system developed at PennApps 2018
        \item Used sentiment analysis, speech to text and NLP to detect bullying
        \item \textbf{Best IoT Hack} and \textbf{Best Education Hack} at PennApps 2018
      \end{itemize}
    \end{tabular} &
    \begin{tabular}[t]{r@{}} \textsc{Sept 2018} \end{tabular}
    \vspace{-0.5em} \\
    % \begin{tabular}[t]{r@{}} \textsc{Aug 2017 - Sept 2018} \end{tabular} &
    % \begin{tabular}[t]{@{}p{36em}}
    % \textbf{RoboBuggy}
    %   \begin{itemize}[leftmargin=*]
    %     \item RoboBuggy is an autonomous competitor in the buggy races at CMU
    %   %   \item Member of infrastructure development team and software representative
    %     \item Used ROS, C++, and Python to build infrastructure to drive the buggy
    %   \end{itemize}
    % \end{tabular}
    % \vspace{-0.5em} \\

    % HackCMU
    \begin{tabular}[t]{@{}p{36em}}
      \textbf{Ugly Duckling} | \textbf{HackCMU 2017}
      \begin{itemize}[leftmargin=*]
        \item An autonomous, deep-learning based robotic camera operator to follow a person using facial recognition
        \item \textbf{Best Use of Machine Learning} by Google, \$1000 cash prize at CMU 50th anniversary celebration, as well as funding from CMU Board of Trustees
      \end{itemize}
    \end{tabular} &
    \begin{tabular}[t]{r@{}} \textsc{Sept 2017} \end{tabular}
    % \vspace{-0.5em} \\
    % \begin{tabular}[t]{r@{}} \textsc{July 2016} \end{tabular} &
    % \begin{tabular}[t]{@{}p{36em}}
    % \textbf{Multi-robot pathfinding} at Stanford University
    %   \begin{itemize}[leftmargin=*]
    %     \item Created a multi-robot pathfinding optimization program at Stanford Pre-Collegiate Summer Institutes - implemented using a coupled approach
    %   \end{itemize}
    % \end{tabular}
    % \vspace{-0.5em} \\

  \end{tabular}
\end{center}

\vspace{-1em}

\section{Technical Skills}
\begin{tabular}{l}
\textbf{Languages:} C/C++, Python, OCaml, Java, Rust, JavaScript, SML, x86/x86-64 \\
\textbf{Applications/Technologies:} Git, Docker, UNIX, Django, Vim, REST, ROS, MySQL, Apache Kafka \\
\end{tabular}

\section{Honors and Achievements}
\begin{tabular}{l}
\textbf{Best IoT Hack, Best Education Hack} awards at PennApps 2018 \\
\textbf{Best Use of Machine Learning} award by Google at HackCMU 2017 \\
\textbf{Top in Country} and \textbf{Top 5 in World} in Cambridge AS and A level Computer Science
\end{tabular}
\end{document}
